\section*{BibTeX Entry Management - Review Duplicate Pairs} % the name of the use case

\paragraph*{Outline} \

\begin{tabular}{ | l | l | }
\hline
Description & User wishes to review a pair of duplicate entries. \\ \hline
Actors & All users with write permissions. \\ \hline
Pre-conditions & The user has performed the Review Duplicates use case and has notified the system that they wish to review a pair of duplicates. \\ \hline
Success Post-conditions & Two duplicate BibTeX entries are shown. \\ \hline
Failure Post-conditions & No duplicates are shown. \\ \hline
\end{tabular}


\paragraph*{Main flow} \

\begin{tabular}{ | l | l | } \hline
1 & System displays a duplicate for the selected entry. \\ \hline
2 & User chooses which of the duplicates to remove. \\ \hline
3 & System confirms deletion of duplicate. \\ \hline
4 & If there is another duplicate of the cite key, the system fetches it and goes back to 2. \\ \hline
5 & If there is no other duplicate of the cite key, the system displays a message to let the user know. \\ \hline
\end{tabular}


\paragraph*{Alternative Flow} \

\begin{tabular}{ | l | l | } \hline
2.a. & System does not contain any duplicates. \\
     & System displays message to user to say that there are no duplicates \\ \hline
\end{tabular}

\paragraph*{Notes} \

A `duplicate' is defined as an entry with the same cite key as another. The set of all duplicates is the set of all entries with one cite key.

