\documentclass{l3proj}
\usepackage{cite}
\usepackage{graphicx}
\usepackage{program}
\usepackage{url}
\usepackage{tabularx}
\usepackage{a4wide}

\begin{document}

\section*{Problem Definition \& Purpose of Project}
BibTeX is a piece of reference and citation management software widely used in conjunction with the LaTeX typesetting tool.  Each reference, or `entry', is stored in a well-defined file format with a set of fields specific to the entry's type. It becomes cumbersome to maintain a collection of entries, particularly when there is the possibility for duplication of entries and entries' identifiers.  This problem is further amplified when groups of people are working on a document and need to be able to maintain a shared library of references.\\
The purpose of the project is to address this problem by providing a web browser-driven tool that allows multiple users to work on sets of shared BibTeX records.

\section*{Functional Requirements}
The tool should provide management of BibTeX records, specifically:
\begin{itemize}
\item Add entry
\item Edit entry
\item View entries
\item Delete entry
\item Undo the deletion of an entry
\item Perform a simple search on entries
\item Perform an advanced search on multiple fields of entries
\item Import entries from a file uploaded by a user
\item Export entries to a file for download by a user
\item Users should be able to group entries to assist with the organisation of citations.
\item Items should expire (deleted items should be removed entirely) after a configurable period
\end{itemize}

\section*{Non-functional Requirements}
The tool should adhere to the following non-functional requirements:
\begin{itemize}
\item Interaction with the system should take place through a web-based interface
\item Entries should be stored in a central database
\item The system should allow multiple users to collaborate on a collection of BibTeX entries.
\item The system should authenticate users and allow various levels of access to users with different privileges.
\item The system should be usable by the target audience, users of the LaTeX typsetting tool.
\item The system should be able to handle concurrent access by users, including 
\end{itemize}

\section*{Use Case Prioritisation}

1 - 3 : Most important to Least important.

1. Add, Edit, Delete, Import Batch, Export All, Basic Search (BibTeX Entry Management Use cases) \\
2. Other BibTeX Entry Management Use Cases, User Management Use Cases \\
3. Advanced Search, Help, Export Current results \& other file formats

Further use cases will be assessed as and when they appear. \\


Risk: Later addition of user and group management might make for difficult domain model modifications later.
Risk Mitigation strategy: Plan the model with future items in mind.



% Bibtex Record Management Use cases
\section*{BibTeX Entry Management - Add Entry} % the name of the use case

\paragraph*{Outline} \

\begin{tabular}{ | l | l | }
\hline
Description & User adds a BibTeX entry to the database. \\ \hline
Actors & All users with write permissions. \\ \hline
Pre-conditions & None. \\ \hline
Success Post-conditions & The entry is added to the database. \\ \hline
Failure Post-conditions & The entry is not added to the database. \\ \hline
\end{tabular}


\paragraph*{Main flow} \

\begin{tabular}{ | l | l | } \hline
1 & User indicates wish to add BibTeX entry. \\ \hline
2 & User Chooses type of BibTeX entry. \\ \hline
3 & System shows relevant fields for current type of entry. \\ \hline
4 & User enters required and optional information for the entry. \\ \hline
5 & User confirms the addition. \\ \hline
6 & System confirms that the information is valid. \\ \hline
7 & System adds the entry to the system and confirms that the addition has been completed. \\ \hline
\end{tabular}


\paragraph*{Alternative Flow} \

\begin{tabular}{ | l | l | } \hline
6.a. & User has input invalid data. \\
     & System raises errors in data and goes back to 4. \\ \hline
6.b. & User has input a cite key that already exists. \\
     & System raises that the cite key already exists and goes back to 4. \\ \hline
\end{tabular}

\paragraph*{Notes} \

At any point between 3 and 5, the user may change the type of BibTeX entry. Any information they have entered shall be transferred to the new entry's fields.


\section*{BibTeX Entry Management - Edit Entry} % the name of the use case

\paragraph*{Outline} \

\begin{tabular}{ | l | l | }
\hline
Description & User edits an existing BibTeX entry and saves changes to the database. \\ \hline
Actors & All users with write permissions. \\ \hline
Pre-conditions & Entry already exists in the database. \\ \hline
Success Post-conditions & The entry is modified in the database. \\ \hline
Failure Post-conditions & The entry is left unmodified in the database in precisely the state. \\
 & it was in before the use case was initiated. \\ \hline
\end{tabular}


\paragraph*{Main flow} \

\begin{tabular}{ | l | l | } \hline
1 & User selects the BibTeX entry to modify. \\ \hline
2 & User amends the entry. \\ \hline
3 & User confirms the modification. \\ \hline
4 & System confirms that the information is valid \& that the cite key is unique. \\ \hline
5 & System confirms that the modification has been completed. \\ \hline
\end{tabular}


\paragraph*{Alternative Flow} \

\begin{tabular}{ | l | l | } \hline
4.a. & User has input invalid data. \\
     & System raises errors in data and goes back to 2. \\ \hline
4.b. & User has input a cite key that already exists. \\
     & System raises that the cite key already exists and goes back to 2 so the user can make changes. \\ \hline
\end{tabular}

\paragraph*{Notes} \

At any point between 2 and 3, the user may change the type of BibTeX entry. Any information they have entered shall be transferred to the new entry's fields.

\paragraph*{Future Amendments} \

Future versions of the program will include group and concurrency functionality -- if a user wishes to modify a `locked' entry, some form of timeout will be initiated in case the first user has forgotten to commit their changes and left the system.


\section*{BibTeX Entry Management - Delete Entry} % the name of the use case

\paragraph*{Outline} \

\begin{tabular}{ | l | l | }
\hline
Description & User deletes an existing BibTeX entry, removing it from the database \\ \hline
Actors & All users (See Future Amendments) \\ \hline
Pre-conditions & None. \\ \hline
Success Post-conditions & The entry is deleted from the database \\ \hline
Failure Post-conditions & The entry is left unmodified in the database in precisely the state \\
 & it was in before the use case was initiated\\ \hline
\end{tabular}


\paragraph*{Main flow} \

\begin{tabular}{ | l | l | } \hline
1 & User selects the BibTeX entry to delete \\ \hline
2 & User presses `delete' \\ \hline
3 & System displays message to confirm the user's intention to delete the entry \\ \hline
4 & System confirms that the modification has been completed \\ \hline
\end{tabular}


\paragraph*{Alternative Flow} \

\begin{tabular}{ | l | l | } \hline
3.a. & User does not want to continue. \\
     & System does not carry out the deletion and goes back to the state before the \\
 & use case was initiated. \\ \hline
\end{tabular}

\paragraph*{Future Amendments} \

In a later iteration, permissions will be added so that some users have access to modify some entries. At this time, the use case will change.


\section*{BibTeX Entry Management - Undo the Deletion of an Entry} % the name of the use case

\paragraph*{Outline} \

\begin{tabular}{ | l | l | }
\hline
Description & User has deleted a BibTeX entry and wants it to be reinstated. \\ \hline
Actors & All users with write permissions. \\ \hline
Pre-conditions & The entry was previously deleted. \\ \hline
Success Post-conditions & The entry is no longer marked as deleted in the database. \\ \hline
Failure Post-conditions & The entry is left unmodified in the database in precisely the state \\
 & it was in before the use case was initiated. \\ \hline
\end{tabular}


\paragraph*{Main flow} \

\begin{tabular}{ | l | l | } \hline
1 & User selects the BibTeX entry to delete from a list of deleted entries. \\ \hline
2 & User presses `Undo delete' \\ \hline
3 & System confirms that the modification has been completed \\ \hline
\end{tabular}

\section*{BibTeX Entry Management - Add Clone of an Entry} % the name of the use case

\paragraph*{Outline} \

\begin{tabular}{ | l | l | }
\hline
Description & User makes an exact clone of an existing BibTeX entry. \\ \hline
Actors & All users with write permissions \\ \hline
Pre-conditions & The target for cloning must exist in the database. \\ \hline
Success Post-conditions & The entry is copied into an `add' window to be modified before addition. \\ \hline
Failure Post-conditions & No clone is added. \\ \hline
\end{tabular}


\paragraph*{Main flow} \

\begin{tabular}{ | l | l | } \hline
1 & User selects the BibTeX entry to clone. \\ \hline
2 & User presses `Add Clone'. \\ \hline
3 & System displays an `Add Entry' dialogue box with the fields from the previous entry displayed. \\ \hline
4 & User confirms that they have finished making modifications. \\ \hline
5 & System confirms that the information added is valid. \\ \hline
6 & System confirms that the modification has been completed. \\ \hline
\end{tabular}


\paragraph*{Alternative Flow} \

\begin{tabular}{ | l | l | } \hline
5.a. & User has input invalid data. \\
     & System raises errors in data and goes back to 3. \\ \hline
5.b. & User has input a cite key that already exists. \\
     & System raises that the cite key already exists and goes back to 3. \\ \hline
\end{tabular}
\section*{BibTeX Entry Management - Review Duplicate Entries} % the name of the use case

\paragraph*{Outline} \

\begin{tabular}{ | l | l | }
\hline
Description & User wishes to view duplicate entries. \\ \hline
Actors & All users. \\ \hline
Pre-conditions & None. \\ \hline
Success Post-conditions & Duplicate BibTeX entries are shown. \\ \hline
Failure Post-conditions & No duplicates are shown. \\ \hline
\end{tabular}


\paragraph*{Main flow} \

\begin{tabular}{ | l | l | } \hline
1 & User selects `Review Duplicates' option. \\ \hline
2 & System collects all entries which are duplicated in the database \\ \hline
3 & System displays list of duplicated entries \\ \hline
\end{tabular}


\paragraph*{Alternative Flow} \

\begin{tabular}{ | l | l | } \hline
2.a. & System does not contain any duplicates. \\
     & System displays message to user to say that there are no duplicates \\ \hline
\end{tabular}

\paragraph*{Notes} \

A `duplicate' is defined as an entry with the same cite key as another. The set of all duplicates is the set of all entries with one cite key.


\section*{BibTeX Entry Management - Review Duplicate Pairs} % the name of the use case

\paragraph*{Outline} \

\begin{tabular}{ | l | l | }
\hline
Description & User wishes to review a pair of duplicate entries. \\ \hline
Actors & All users with write permissions. \\ \hline
Pre-conditions & The user has performed the Review Duplicates use case and has notified the system that they wish to review a pair of duplicates. \\ \hline
Success Post-conditions & Two duplicate BibTeX entries are shown. \\ \hline
Failure Post-conditions & No duplicates are shown. \\ \hline
\end{tabular}


\paragraph*{Main flow} \

\begin{tabular}{ | l | l | } \hline
1 & System displays a duplicate for the selected entry. \\ \hline
2 & User chooses which of the duplicates to remove. \\ \hline
3 & System confirms deletion of duplicate. \\ \hline
4 & If there is another duplicate of the cite key, the system fetches it and goes back to 2. \\ \hline
5 & If there is no other duplicate of the cite key, the system displays a message to let the user know. \\ \hline
\end{tabular}


\paragraph*{Alternative Flow} \

\begin{tabular}{ | l | l | } \hline
2.a. & System does not contain any duplicates. \\
     & System displays message to user to say that there are no duplicates \\ \hline
\end{tabular}

\paragraph*{Notes} \

A `duplicate' is defined as an entry with the same cite key as another. The set of all duplicates is the set of all entries with one cite key.


\section*{BibTeX Entry Management - Import Batch}

\paragraph*{Outline} \

\begin{tabular}{ | l | l | }
\hline
Description & User imports BibTeX entries to the database by uploading a batch file. \\ \hline
Actors & All users with write permissions. \\ \hline
Pre-conditions & The file exists on the User's system. \\ \hline
Success Post-conditions & All entries that do not create exact duplicates are added to the database. \\ \hline
Failure Post-conditions & No entries are added to the database. \\ \hline
\end{tabular}


\paragraph*{Main flow} \

\begin{tabular}{ | l | l | } \hline
1 & User notifies the system that they wish to upload a BibTeX batch file. \\ \hline
2 & System displays the import file screen. \\ \hline
3 & User selects the file for upload on their computer. \\ \hline
4 & User uploads the file for import. \\ \hline
5 & System parses the file. \\ \hline
6 & System performs the `Add Entry' use case (Steps 6 \& 7) for all entries. \\ \hline
7 & System confirms that the modifications have been completed. \\ \hline
\end{tabular}


\paragraph*{Alternative Flow} \

\begin{tabular}{ | l | l | } \hline
3.a	& User has selected an unsupported file. \\
	& System rejects the file, raising errors and goes back to 2. \\ \hline
4.a	& The filesize is too large. (Maximum size TBA.) \\
	& System rejects the file, raising errors and goes back to 2. \\ \hline
5.a	& The file is in an invalid format. \\
	& System rejects the file, raising errors and goes back to 2. \\ \hline
5.a	& The file cannot be parsed. \\
	& System rejects the file, raising errors and goes back to 2. \\ \hline
6.a. & User has invalid data in the file. \\
     & System raises errors in data, rejects the file and goes back to 2. \\ 
     & Rationale: The action is intended to be atomic; if any resubmission \\
	& has to happen this will avoid duplicates \\ \hline
6.b. & Entry has an existing (exact) match in the database. \\
     & System ignores the duplicate and moves onto the next entry. \\ 
     & Rationale: Reduce the number of duplicates in the system. \\ \hline
\end{tabular}

\paragraph*{Notes} \

Maximum filesize has yet to be agreed.
This use case is intended to be atomic; if any resubmission has to happen this will avoid unnecessary duplicates.

\paragraph*{Future Amendments} \

Initially, the system will only be able to handle .bib files. This will most likely be expanded in a later iteration.


\section*{BibTeX Entry Management - Export Current Query Results}

\paragraph*{Outline} \

\begin{tabular}{ | l | l | }
\hline
Description & User exports the BibTeX entries they can see \\ \hline
Actors & All users (See  Future Amendments) \\ \hline
Pre-conditions & The user has some results to export. \\ \hline
Success Post-conditions & A file is downloaded by the user. \\ \hline
Failure Post-conditions & No file is downloaded by the user. \\ \hline
\end{tabular}


\paragraph*{Main flow} \

\begin{tabular}{ | l | l | } \hline
1 & User notifies the system that they wish to download a BibTeX file of the current results. \\ \hline
2 & System displays a file export screen. \\ \hline
3 & System generates the file from the current query data \\ \hline
4 & User downloads the file. \\ \hline
5 & User is redirected to the previous (results) page. \\ \hline
\end{tabular}


\paragraph*{Alternative Flow} \

No anticipated alternative flow.  Option for export will not be displayed if there are no current query results.

\paragraph*{Future Amendments} \

Initially, the system will only be able to handle .bib files. This will most likely be expanded in a later iteration.


\section*{BibTeX Entry Management - Export All Entries}

\paragraph*{Outline} \

\begin{tabular}{ | l | l | }
\hline
Description & User exports all BibTeX entries in the database. \\ \hline
Actors & All users (See  Future Amendments) \\ \hline
Pre-conditions & There are entries in the database. \\ \hline
Success Post-conditions & A file is downloaded by the user. \\ \hline
Failure Post-conditions & No file is downloaded by the user. \\ \hline
\end{tabular}


\paragraph*{Main flow} \

\begin{tabular}{ | l | l | } \hline
1 & User notifies the system that they wish to download a BibTeX file of all items in the database. \\ \hline
2 & System displays a file export screen. \\ \hline
3 & System generates the file from all BibTeX entry data. \\ \hline
4 & User downloads the file. \\ \hline
5 & User is redirected to the `home' view. \\ \hline
\end{tabular}


\paragraph*{Alternative Flow} \

No anticipated alternative flow.  Option for export will not be displayed if there are no current entries in the database.

\paragraph*{Future Amendments} \

Initially, the system will only be able to handle .bib files. This will most likely be expanded in a later iteration.



% System Tools Use Cases
\section*{System Tools - Basic Search} % the name of the use case

\paragraph*{Outline} \

\begin{tabular}{ | l | l | }
\hline
Description & User searches BibTex entries in the database. \\ \hline
Actors & All users \\ \hline
Pre-conditions & None. \\ \hline
Success Post-conditions & Results of the search are displayed \\ \hline
\end{tabular}


\paragraph*{Main flow} \

\begin{tabular}{ | l | l | } \hline
1 & User types into search box and chooses which field to search on. \\ \hline
2 & System updates results pane with each keypress or field choice update and goes back to 1. \\ \hline
\end{tabular}


\paragraph*{Alternative Flow} \

If there are no search results for any query then the system will display a message to bring it to the user's attention.

\paragraph*{Notes} \

Searches will be conducted on the field selected at the time.

\paragraph*{Future Amendments} \

The initial implementation of search may not be `live-updating' to decrease initial development time. \\
A search on which user entered or uploaded the entry may be relevant. \\
A search across multiple fields may be relevant


\section*{System Tools - Help} % the name of the use case

\paragraph*{Outline} \

\begin{tabular}{ | l | l | }
\hline
Description & User would like help in using the system. \\ \hline
Actors & All users \\ \hline
\end{tabular}

\paragraph*{Notes} \

Help topics should be available for each use case. Precise format yet to be decided, possibilities include:

\begin{itemize}
\item Wiki
\item HTML-based help system
\item FAQ list
\end{itemize}




% Group Support Use Cases
\section*{Group Management - Create Group} % the name of the use case

\paragraph*{Outline} \

\begin{tabular}{ | l | l | }
\hline
Description & User creates a user group \\ \hline
Actors & All users with group creation permissions \\ \hline
Pre-conditions & The group does not already exist. \\ \hline
Success Post-conditions & The group is created in the database \\ \hline
Failure Post-conditions & The group is not created in the database \\ \hline
\end{tabular}


\paragraph*{Main flow} \

\begin{tabular}{ | l | l | } \hline
1 & User indicates wish to create a group. \\ \hline
2 & System shows create group interface \\ \hline
3 & User enters information for the group \\ \hline
4 & User confirms the addition\\ \hline
5 & System confirms that the group does not already exist \\ \hline
6 & System adds the group to the system and confirms that the addition has been completed\\ \hline
\end{tabular}


\paragraph*{Alternative Flow} \

\begin{tabular}{ | l | l | } \hline
5.a. & Group already exists. \\
     & System raises errors in data and goes back to 2, with information displayed in fields.\\ \hline
\end{tabular}

\paragraph*{Notes} \

The groups functionality may not be included until later stages of development. \\
Groups will be unique based on their names, not case sensitive.

\section*{Group Management - Edit Group Information} % the name of the use case

\paragraph*{Outline} \

\begin{tabular}{ | l | l | }
\hline
Description & User amend details (name and description only) stored about a user group \\ \hline
Actors & The creator of the group. \\ \hline
Pre-conditions & The group already exists and the current user created the group. \\ \hline
Success Post-conditions & The group is modified in the database. \\ \hline
Failure Post-conditions & The group is not modified in the database. \\ \hline
\end{tabular}


\paragraph*{Main flow} \

\begin{tabular}{ | l | l | } \hline
1 & User indicates wish to edit a group by selecting it and choosing the edit option. \\ \hline
2 & System shows edit group interface. \\ \hline
3 & User edits the information for the group \\ \hline
4 & User confirms the modification. \\ \hline
5 & System confirms that the modification does not attempt to overwrite an existing group. \\ \hline
6 & System modifies the group and confirms that the modification has been completed. \\ \hline
\end{tabular}


\paragraph*{Alternative Flow} \

\begin{tabular}{ | l | l | } \hline
5.a. & Changes will overwrite a group. \\
     & System raises that the data causes an error and goes back to 2, with information  \\
  & displayed in fields. \\ \hline
\end{tabular}

\paragraph*{Notes} \

The groups functionality may not be included until later stages of development. \\
Groups will be unique based on their names, not case sensitive.

\section*{Group Management - Disband Group} % the name of the use case

\paragraph*{Outline} \

\begin{tabular}{ | l | l | }
\hline
Description & User deletes a user group. \\ \hline
Actors & The creator of the group. \\ \hline
Pre-conditions & The group already exists and the current user created the group. \\ \hline
Success Post-conditions & The group is no longer displayed in the system. \\ \hline
Failure Post-conditions & The group is still displayed in the system. \\ \hline
\end{tabular}


\paragraph*{Main flow} \

\begin{tabular}{ | l | l | } \hline
1 & User indicates wish to disband a group by selecting it and choosing the disband option. \\ \hline
2 & System shows disband group interface. \\ \hline
3 & User confirms the removal. \\ \hline
4 & System modifies the group and confirms that the modification has been completed. \\ \hline
\end{tabular}


\paragraph*{Alternative Flow} \

\begin{tabular}{ | l | l | } \hline
5.a. & User does not want to disband the group. \\
     & System goes back to before the use case, having made no modifications to the system.\\ \hline
\end{tabular}

\paragraph*{Notes} \

The groups functionality may not be included until later stages of development.

\section*{Group Management - Request to Join Group} % the name of the use case

\paragraph*{Outline} \

\begin{tabular}{ | l | l | }
\hline
Description & User requests to join a user group \\ \hline
Actors & All users with write permissions \\ \hline
Pre-conditions & The user is not already in the group they wish to join and the group exists. \\ \hline
Success Post-conditions & The user invokes a request to join the group. \\ \hline
Failure Post-conditions & The user does not invoke a request to join the group. \\ \hline
\end{tabular}


\paragraph*{Main flow} \

\begin{tabular}{ | l | l | } \hline
1 & User indicates wish to join a group. \\ \hline
2 & If the group is private, the system adds the user to an approval list, otherwise go to 3. \\ \hline
3 & If the group is public, the system adds the user to the group member list. \\ \hline
\end{tabular}

\paragraph*{Notes} \

The groups functionality may not be included until later stages of development.

\section*{Group Management - Review Group Membership} % the name of the use case

\paragraph*{Outline} \

\begin{tabular}{ | l | l | }
\hline
Description & User reviews their group membership status. \\ \hline
Actors & All users. \\ \hline
Pre-conditions & The user is logged in. \\ \hline
Success Post-conditions & The system displays a user's group membership status. \\ \hline
Failure Post-conditions & The system does not display a user's group membership status. \\ \hline
\end{tabular}


\paragraph*{Main flow} \

\begin{tabular}{ | l | l | } \hline
1 & User indicates wish to view group memberships. \\ \hline
2 & System collects information about the user's groups. \\ \hline
3 & System displays user's group information. \\ \hline
\end{tabular}


\paragraph*{Alternative Flow} \

\begin{tabular}{ | l | l | } \hline
2.a. & User is not in any groups. \\
     & System notifies user that they are not a member of any groups. \\ \hline
\end{tabular}

\paragraph*{Notes} \

The groups functionality may not be included until later stages of development. \\
Groups will be unique based on their names, not case sensitive.

\section*{Group Management - Review Requests to Join Group} % the name of the use case

\paragraph*{Outline} \

\begin{tabular}{ | l | l | }
\hline
Description & User wants to review requests to join one of their user groups. \\ \hline
Actors & The creator of the group \\ \hline
Pre-conditions & There are requests to join (one of) the user's groups. \\ \hline
Success Post-conditions & The user views the requests to join one of their groups. \\ \hline
Failure Post-conditions & The user does not view requests. \\ \hline
\end{tabular}


\paragraph*{Main flow} \

\begin{tabular}{ | l | l | } \hline
1 & User indicates wish to review group join requests. \\ \hline
2 & User clicks on accept or reject for each entry. \\ \hline
3 & System adds the requestee to the group or rejects their request, depending on 2. \\ \hline
\end{tabular}


\paragraph*{Alternative flow} \

\begin{tabular}{ | l | l | } \hline
2.a & User wants to deal with a request later \\ 
	& System makes no modification to the requests not yet dealt with \\ \hline
\end{tabular}

\paragraph*{Notes} \

The groups functionality may not be included until later stages of development.



% User Management Use Cases
\section*{BibTeX Entry Management - Alter User Permissions} % the name of the use case

\paragraph*{Outline} \

\begin{tabular}{ | l | l | }
\hline
Description & Administrator wants to modify a user's permissions. \\ \hline
Actors & Administrator. \\ \hline
Pre-conditions & The user they wish to modify exists in the system exists in the system and was selected in the `Manage Users' use case. \\ \hline
Success Post-conditions & The user's permissions are altered in the database. \\ \hline
Failure Post-conditions & The user's permissions are not altered. \\ \hline
\end{tabular}


\paragraph*{Main flow} \

\begin{tabular}{ | l | l | } \hline
1 & Administrator chooses which permissions to assign to a user. \\ \hline
2 & Administrator confirms that they have selected the correct permissions. \\ \hline
3 & System performs modifications in the database. \\ \hline
4 & System confirms the modification. \\ \hline
\end{tabular}
\section*{User Management - New User Sign Up} % the name of the use case

\paragraph*{Outline} \

\begin{tabular}{ | l | l | }
\hline
Description & New User registers with the system. \\ \hline
Actors & Unregistered users. \\ \hline
Pre-conditions & Email address has not been used before to sign up. \\ \hline
Success Post-conditions & Confirmation email sent to provided email address. \\ \hline
Failure Post-conditions & Confirmation email not sent to provided email address. \\ \hline
\end{tabular}


\paragraph*{Main flow} \

\begin{tabular}{ | l | l | } \hline
1 & User indicates wish to sign up to the system. \\ \hline
2 & User enters valid email address. \\ \hline
3 & User enters given and family names. \\ \hline
4 & User confirms their details are correct. \\ \hline
5 & System confirms that the information is valid. \\ \hline
6 & System sends confirmation code to the provided \\
  & email address and notifies the user to check their email inbox. \\ \hline
\end{tabular}


\paragraph*{Alternative Flow} \

\begin{tabular}{ | l | l | } \hline
2.a. & User has input an invalid email address. \\
     & System raises errors when the user moves  \\
  & onto the next field and goes back to 2 when they re-enter their email address. \\ \hline
5.b. & User has input an email address that is already in use. \\
     & System raises that an account for that email address  \\
  & already exists and goes back to 2 with fields pre-populated. \\ \hline
\end{tabular}
\section*{User Management - Log In} % the name of the use case

\paragraph*{Outline} \

\begin{tabular}{ | l | l | }
\hline
Description & User logs in to the system. \\ \hline
Actors & All users who have signed up. \\ \hline
Pre-conditions & The user has an account and is not already logged in. \\ \hline
Success Post-conditions & The user is logged in to the system. \\ \hline
Failure Post-conditions & The user is not logged in to the system. \\ \hline
\end{tabular}


\paragraph*{Main flow} \

\begin{tabular}{ | l | l | } \hline
1 & User chooses `log in'. \\ \hline
2 & User enters username and password. \\ \hline
3 & System ensures that the user has not been issued with a temporary password. \\ \hline
4 & User is logged in to the system. \\ \hline
\end{tabular}


\paragraph*{Alternative Flow} \

\begin{tabular}{ | l | l | } \hline
3.a. & User is logging in with a temporary password. \\
     & System displays the `change password' use case. \\
     & When the change password use case is completed,  \\
  & the user is logged in and can begin using the system.  \\
	 & If they fail to change their password they are not logged in \\ \hline
\end{tabular}
\section*{User Management - Log Out} % the name of the use case

\paragraph*{Outline} \

\begin{tabular}{ | l | l | }
\hline
Description & User is finished using the system and wants to log out. \\ \hline
Actors & All users. \\ \hline
Pre-conditions & The user is logged in. \\ \hline
Success Post-conditions & The user is logged out of the system. \\ \hline
Failure Post-conditions & The user is logged out of the system. \\ \hline
\end{tabular}


\paragraph*{Main flow} \

\begin{tabular}{ | l | l | } \hline
1 & User indicates wish to log out. \\ \hline
2 & System logs the user out. \\ \hline
\end{tabular}


\paragraph*{Alternative Flow} \

If the user is not logged in, the log in page will be displayed.

\paragraph*{Notes} \

The user is logged out of the system regardless of failure or success of the use case to ensure that their account is not left logged in after they press `Log out'.


\section*{User Management - View All Users} % the name of the use case

\paragraph*{Outline} \

\begin{tabular}{ | l | l | }
\hline
Description & Administrator wants to view all users. \\ \hline
Actors & Administrator. \\ \hline
Pre-conditions & None. \\ \hline
Success Post-conditions & The user list is shown to the administrator. \\ \hline
Failure Post-conditions & The user list is not shown. \\ \hline
\end{tabular}


\paragraph*{Main flow} \

\begin{tabular}{ | l | l | } \hline
1 & Administrator indicates wish to view all users. \\ \hline
2 & System collects all users and displays them to the administrator. \\ \hline
3 & Administrator can choose to upgrade or downgrade a user's privileges. \\ \hline
\end{tabular}
\section*{User Management - New User Sign Up} % the name of the use case

\paragraph*{Outline} \

\begin{tabular}{ | l | l | }
\hline
Description & New User registers with the system. \\ \hline
Actors & Unregistered users. \\ \hline
Pre-conditions & Email address has not been used before to sign up. \\ \hline
Success Post-conditions & Confirmation email sent to provided email address. \\ \hline
Failure Post-conditions & Confirmation email not sent to provided email address. \\ \hline
\end{tabular}


\paragraph*{Main flow} \

\begin{tabular}{ | l | l | } \hline
1 & User indicates wish to sign up to the system. \\ \hline
2 & User enters valid email address. \\ \hline
3 & User enters given and family names. \\ \hline
4 & User confirms their details are correct. \\ \hline
5 & System confirms that the information is valid. \\ \hline
6 & System sends confirmation code to the provided \\
  & email address and notifies the user to check their email inbox. \\ \hline
\end{tabular}


\paragraph*{Alternative Flow} \

\begin{tabular}{ | l | l | } \hline
2.a. & User has input an invalid email address. \\
     & System raises errors when the user moves  \\
  & onto the next field and goes back to 2 when they re-enter their email address. \\ \hline
5.b. & User has input an email address that is already in use. \\
     & System raises that an account for that email address  \\
  & already exists and goes back to 2 with fields pre-populated. \\ \hline
\end{tabular}
\section*{User Management - Reset Password} % the name of the use case

\paragraph*{Outline} \

\begin{tabular}{ | l | l | }
\hline
Description & User has forgotten their password and wants to regain access to  \\
  & their account. \\ \hline
Actors & All users with an account. \\ \hline
Pre-conditions & The user has completed the sign up process. \\ \hline
Success Post-conditions & The system sends a temporary password to their email address. \\ \hline
Failure Post-conditions & The system does not send a temporary password. \\ \hline
\end{tabular}


\paragraph*{Main flow} \

\begin{tabular}{ | l | l | } \hline
1 & User indicates that they have forgotten their password. \\ \hline
2 & System sends temporary generated password to the user's  \\
  & email address and changes their password to match. \\ \hline
3 & System tells user to check their email. \\ \hline
\end{tabular}


\end{document}
