\section*{BibTeX Entry Management - Add Entry} % the name of the use case

\paragraph*{Outline} \

\begin{tabular}{ | l | l | }
\hline
Description & User adds a BibTeX entry to the database. \\ \hline
Actors & All users with write permissions. \\ \hline
Pre-conditions & None. \\ \hline
Success Post-conditions & The entry is added to the database. \\ \hline
Failure Post-conditions & The entry is not added to the database. \\ \hline
\end{tabular}


\paragraph*{Main flow} \

\begin{tabular}{ | l | l | } \hline
1 & User indicates wish to add BibTeX entry. \\ \hline
2 & User Chooses type of BibTeX entry. \\ \hline
3 & System shows relevant fields for current type of entry. \\ \hline
4 & User enters required and optional information for the entry. \\ \hline
5 & User confirms the addition. \\ \hline
6 & System confirms that the information is valid. \\ \hline
7 & System adds the entry to the system and confirms that the addition has been completed. \\ \hline
\end{tabular}


\paragraph*{Alternative Flow} \

\begin{tabular}{ | l | l | } \hline
6.a. & User has input invalid data. \\
     & System raises errors in data and goes back to 4. \\ \hline
6.b. & User has input a cite key that already exists. \\
     & System raises that the cite key already exists and goes back to 4. \\ \hline
\end{tabular}

\paragraph*{Notes} \

At any point between 3 and 5, the user may change the type of BibTeX entry. Any information they have entered shall be transferred to the new entry's fields.

