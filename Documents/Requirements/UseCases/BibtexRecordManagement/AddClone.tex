\section*{BibTeX Entry Management - Add Clone of an Entry} % the name of the use case

\paragraph*{Outline} \

\begin{tabular}{ | l | l | }
\hline
Description & User makes an exact clone of an existing BibTeX entry. \\ \hline
Actors & All users with write permissions \\ \hline
Pre-conditions & The target for cloning must exist in the database. \\ \hline
Success Post-conditions & The entry is copied into an `add' window to be modified before addition. \\ \hline
Failure Post-conditions & No clone is added. \\ \hline
\end{tabular}


\paragraph*{Main flow} \

\begin{tabular}{ | l | l | } \hline
1 & User selects the BibTeX entry to clone. \\ \hline
2 & User presses `Add Clone'. \\ \hline
3 & System displays an `Add Entry' dialogue box with the fields from the previous entry displayed. \\ \hline
4 & User confirms that they have finished making modifications. \\ \hline
5 & System confirms that the information added is valid. \\ \hline
6 & System confirms that the modification has been completed. \\ \hline
\end{tabular}


\paragraph*{Alternative Flow} \

\begin{tabular}{ | l | l | } \hline
5.a. & User has input invalid data. \\
     & System raises errors in data and goes back to 3. \\ \hline
5.b. & User has input a cite key that already exists. \\
     & System raises that the cite key already exists and goes back to 3. \\ \hline
\end{tabular}