\section*{BibTeX Entry Management - Delete Entry} % the name of the use case

\paragraph*{Outline} \

\begin{tabular}{ | l | l | }
\hline
Description & User deletes an existing BibTeX entry, removing it from the database \\ \hline
Actors & All users (See Future Amendments) \\ \hline
Pre-conditions & None. \\ \hline
Success Post-conditions & The entry is deleted from the database \\ \hline
Failure Post-conditions & The entry is left unmodified in the database in precisely the state \\
 & it was in before the use case was initiated\\ \hline
\end{tabular}


\paragraph*{Main flow} \

\begin{tabular}{ | l | l | } \hline
1 & User selects the BibTeX entry to delete \\ \hline
2 & User presses `delete' \\ \hline
3 & System displays message to confirm the user's intention to delete the entry \\ \hline
4 & System confirms that the modification has been completed \\ \hline
\end{tabular}


\paragraph*{Alternative Flow} \

\begin{tabular}{ | l | l | } \hline
3.a. & User does not want to continue. \\
     & System does not carry out the deletion and goes back to the state before the \\
 & use case was initiated. \\ \hline
\end{tabular}

\paragraph*{Future Amendments} \

In a later iteration, permissions will be added so that some users have access to modify some entries. At this time, the use case will change.

