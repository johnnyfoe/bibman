\chapter{Conclusion}
\label{conclusion}
The conclusion chapter of this dissertation draws to a close the discussion of the project.

It performs a retrospective assessment of the successes and shortcomings of the project with respect to the aims and identifies some future work for the area.

Finally, it analyses the personal development that the author went through during the project and some of the lessons learned about single-developer projects.

\section{Summary}
Upon reflection on the aims of the project, the author feels that the final year undertaking was a success.

It is felt that a far better understanding of the technologies used has been gained, particularly in the areas of NHibernate and \gls{asp}\gls{net}.  

It is also felt that good software engineering practice was observed during development and testing.  With that said, it was realised during development of the system that some of the practice employed turned out to be more of a bureaucratic and unnecessary measure than an assistive process.  Common sense prevailed in these scenarios, and were particularly noticeable by the scaling back of unit testing and \gls{ci}.  

Some software engineering practices were helpful.  The bug tracking software was especially helpful having been employed with a common-sense approach in mind.  Its greatest benefit was that it ensured that defects with the software did not go unnoticed or unresolved.

The professional attitude that has been developed by the author over approximately six months has been career-changing for the better.  The project required deep knowledge and understanding at every stage of the project.  The author had to perpetually be on top form; had he not been, he would not have been exposed to as many important concepts, would not have learned as much and would not have produced as usable or useful a system as was done.

The software solution produced was found to be a success --- an outcome determined by statistical analysis of evaluation participants� responses to a questionnaire. The measurements derived from this showed that all participants reacted positively to questions on the usability of the software solution.  Moreover, they encouraged further development of the solution and even suggested specific areas of further work.

It might work in the developer's favour to alert the reader to the fact that the bibliography for this dissertation was managed entirely by the software product produced this year.  

\section{Suggestions for Further Work}
Suggestions of areas for further work on this software solution include:
\begin{enumerate}
	\item Allow the tagging of entries (as set out in the Discovery model); 
	\item Implement a help system for users;
	\item Allow users to upload to a tag;
	\item Allow users to download a tag or specific set of entries;
	\item Allow the grouping of users with privileges attached to their roles;
	\item Use a push-based approach for concurrent access rather than polling the server.  Many redundant packets are sent and received because of the approach taken at the moment.  It would be better in terms of performance, server load and scalability if the system did not have to frequently respond with empty replies;
	\item Use a more rigorous approach to authentication.
\end{enumerate}

\section{Personal Lessons and Professional Development}
The author of this dissertation underwent substantial personal development during the production of this software solution, particularly in terms of technical knowledge gained, professional conduct and software engineering discipline gained.

It was also reassuring to confirm that the author's choice of software engineering as a career is an absolutely suitable and exciting path, having thoroughly enjoyed the challenge of working with Dr Manlove to design, produce, evaluate and discuss the software solution offered by the project.
