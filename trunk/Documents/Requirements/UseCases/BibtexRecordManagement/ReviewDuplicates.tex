\section*{BibTeX Entry Management - Review Duplicate Entries} % the name of the use case

\paragraph*{Outline} \

\begin{tabular}{ | l | l | }
\hline
Description & User wishes to view duplicate entries. \\ \hline
Actors & All users. \\ \hline
Pre-conditions & None. \\ \hline
Success Post-conditions & Duplicate BibTeX entries are shown. \\ \hline
Failure Post-conditions & No duplicates are shown. \\ \hline
\end{tabular}


\paragraph*{Main flow} \

\begin{tabular}{ | l | l | } \hline
1 & User selects `Review Duplicates' option. \\ \hline
2 & System collects all entries which are duplicated in the database \\ \hline
3 & System displays list of duplicated entries \\ \hline
\end{tabular}


\paragraph*{Alternative Flow} \

\begin{tabular}{ | l | l | } \hline
2.a. & System does not contain any duplicates. \\
     & System displays message to user to say that there are no duplicates \\ \hline
\end{tabular}

\paragraph*{Notes} \

A `duplicate' is defined as an entry with the same cite key as another. The set of all duplicates is the set of all entries with one cite key.

