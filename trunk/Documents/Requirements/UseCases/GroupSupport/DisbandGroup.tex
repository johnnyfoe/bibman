\section*{Group Management - Disband Group} % the name of the use case

\paragraph*{Outline} \

\begin{tabular}{ | l | l | }
\hline
Description & User deletes a user group. \\ \hline
Actors & The creator of the group. \\ \hline
Pre-conditions & The group already exists and the current user created the group. \\ \hline
Success Post-conditions & The group is no longer displayed in the system. \\ \hline
Failure Post-conditions & The group is still displayed in the system. \\ \hline
\end{tabular}


\paragraph*{Main flow} \

\begin{tabular}{ | l | l | } \hline
1 & User indicates wish to disband a group by selecting it and choosing the disband option. \\ \hline
2 & System shows disband group interface. \\ \hline
3 & User confirms the removal. \\ \hline
4 & System modifies the group and confirms that the modification has been completed. \\ \hline
\end{tabular}


\paragraph*{Alternative Flow} \

\begin{tabular}{ | l | l | } \hline
5.a. & User does not want to disband the group. \\
     & System goes back to before the use case, having made no modifications to the system.\\ \hline
\end{tabular}

\paragraph*{Notes} \

The groups functionality may not be included until later stages of development.
